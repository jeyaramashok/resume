\documentclass[a4paper,11pt]{article}

\usepackage{latexsym}
\usepackage[empty]{fullpage}
\usepackage{titlesec}
\usepackage{marvosym}
\usepackage[usenames,dvipsnames]{color}
\usepackage{verbatim}
\usepackage{enumitem}
\usepackage[hidelinks]{hyperref}
\usepackage{fancyhdr}
\usepackage[english]{babel}
\usepackage{tabularx}
\usepackage{lastpage}
\usepackage{fontawesome}
\usepackage[margin=3cm]{geometry}

\fancyhf{} % clear all header and footer fields
\fancyfoot{}
\renewcommand{\headrulewidth}{0pt}
\renewcommand{\footrulewidth}{0pt}
\rfoot{Page \thepage \hspace{1pt} of \pageref{LastPage}}

% Adjust margins
\addtolength{\oddsidemargin}{-0.5in}
\addtolength{\evensidemargin}{-0.5in}
\addtolength{\textwidth}{1in}
\addtolength{\topmargin}{-.5in}
\addtolength{\textheight}{1.0in}
\urlstyle{same}

\raggedbottom
\raggedright
\setlength{\tabcolsep}{0in}

% Sections formatting
\titleformat{\section}{
  \vspace{-4pt}\scshape\raggedright\large
}{}{0em}{}[\color{black}\titlerule \vspace{-5pt}]

%-------------------------
% Custom commands
\newcommand{\resumeItem}[2]{
  \item[]\small{
    \textbf{#1}{: #2 \vspace{-2pt}}
  }
}


\newcommand{\resumeSubheading}[4]{
  \vspace{-1pt}\item[]
    \begin{tabular*}{0.97\textwidth}[t]{l@{\extracolsep{\fill}}r}
      \textbf{#1} & #2 \\
      \textit{\small#3} & \textit{\small #4} \\
    \end{tabular*}\vspace{-5pt}
}

\newcommand{\resumeSubSubheading}[2]{
    \begin{tabular*}{0.97\textwidth}{l@{\extracolsep{\fill}}r}
      \textit{\small#1} & \textit{\small #2} \\
    \end{tabular*}\vspace{-5pt}
}

\newcommand{\resumeSubItem}[2]{\resumeItem{#1}{#2}\vspace{-4pt}}

\renewcommand{\labelitemii}{$\circ$}

\newcommand{\resumeSubHeadingListStart}{\begin{itemize}[leftmargin=*]}
\newcommand{\resumeSubHeadingListEnd}{\end{itemize}}
\newcommand{\resumeItemListStart}{\begin{itemize}[leftmargin=*]}
\newcommand{\resumeItemListEnd}{\end{itemize}\vspace{-5pt}}

%-------------------------------------------
%%%%%%  CV STARTS HERE  %%%%%%%%%%%%%%%%%%%%%%%%%%%%

\begin{document}

%----------HEADING-----------------
\begin{tabular*}{\textwidth}{l@{\extracolsep{\fill}}r}
    \textbf{\href{https://github.com/jeyaramashok}{\Large Jeyaram Ashokraj}} & Email : \href{mailto:a.jeyaram@gmail.com}{a.jeyaram@gmail.com}\\
\end{tabular*}

%-----------SUMMARY-----------------
\section{Summary}

I am a software engineer at IBM where I build cloud native applications and help product teams to adopt best practices for developing production grade applications.

%-----------EXPERIENCE-----------------
\section{Experience}
\resumeSubHeadingListStart

\resumeSubheading
{IBM}{Rochester, MN, USA}
{Software Engineer}{Oct 2015 - Present}
\resumeItemListStart
\resumeItem{CloudPak Engineering}
{
    \begin{itemize}[leftmargin=*]
        \item Responsible for helping and guiding product teams to pass internal IBM certification for developing applications on Openshift and Kubernetes.
        \item Developed kubernetes operator for redis database and service broker (implementing OSB API).
        \item Responsible for maintaining inner source components.
        \item Developed certification checklist containing security best practices and patterns that IBM products must adhere to when developing production grade applications on Kubernetes.
        \item Developed solutions for disconnected installs of cloudpak products.
        \item Developed internal tools (CLI) to improve the developer experience for packaging and delivering products.
        \item Helped onboarding ISV and opensource helm charts into IBM catalog by addressing security gaps and hardening things to help them meet IBM standards for Kubernetes software.
        \item Contributed to opensource projects like operator-sdk, operator-lifecycle-manager.
    \end{itemize}
}
\resumeItem{Watson Natural Language Understanding}
{
    \begin{itemize}[leftmargin=*]
        \item An API as a service platform for natural language understanding tasks.
        \item Responsible for engineering, implementation, monitoring, and maintenance of the service.
        \item Integrated sentiment service with existing stack (implemented in Typescript).
        \item Developed helm charts to deploy the product in IBM Private cloud.
        \item Created scripts to automate migration of standalone databases to IBM Cloud Database instances.
    \end{itemize}
}

\resumeItem{Analytics Engine}
{
    \begin{itemize}[leftmargin=*]
        \item A IaaS compute platform running Apache Hadoop and Spark.
        \item Worked on CLI component to interact with the cluster, launch spark jobs and retrieve logs.
        \item Implemented the webHDFS REST api as file system commands in the CLI.
    \end{itemize}
}

\resumeItem{BigInsights on Cloud}
{
    \begin{itemize}[leftmargin=*]
        \item A big data platform running Apache Hadoop on VM's and Baremetal machines.
        \item Developed Chef recipes to scale clusters by adding nodes, backup and restore.
        \item Parallelized delivery of security fixes to clusters.
        \item Worked on making the service GDPR complaint by adding Vault support and encrypting disks.
        \item Encryption was challenging task due to large size of disks ( 4 TB X 8 disk X 5-8 nodes). It's typically done by backing up data to temp disk storage and then encrypting it, but instead I proposed to use Hadoop’s self-healing and rack-awareness to handle data loss during encryption which cut down time from weeks to days.
    \end{itemize}
}

\resumeItem{Spark as a Service}
{
    \begin{itemize}[leftmargin=*]
        \item A multi-tenant platform for running notebooks and batch jobs with Apache Spark, running on top IBM Spectrum and Apache Mesos.
        \item Reponsible for engineering, implementation, monitoring, and maintenance of the service.
        \item Created a containerized integration test framework using Cucumber for the CICD pipeline.
        \item Improved the platform security by fixing vulnerabilities identified from external penetration testing.
    \end{itemize}
}

\resumeItem{Analytics NextGen Workbench}
{
    \begin{itemize}[leftmargin=*]
        \item A platform for data scientists to design and develop predictive models and execute with SPSS backend.
        \item I was responsible for developing a scheduler microservice using Akka/Scala for the platform.
    \end{itemize}
}

\resumeItemListEnd

\newpage

\resumeSubheading
{University of South Florida}{Tampa, FL, US}
{Research and Teaching Assistant}{Aug 2014 - May 2015}
\resumeItemListStart
\resumeItem{Research Assistant}
{
    \begin{itemize}[leftmargin=*]
        \item Analyzed customer software subscriptions data provided by Wharton customer analytics initiative (wcai) research group.
        \item Involved in cleaning and preparation of data set, feature extraction, data visualization and identifying research questions.
    \end{itemize}
}
\resumeItem{Teaching Assistant}
{
    \begin{itemize}[leftmargin=*]
        \item TA for two graduate level courses: Distributed Information Systems course (ISM 6225) and Information Security and
              Risk Management course (ISM 6328)
    \end{itemize}
}
\resumeItem{Software Developer}
{
    \begin{itemize}[leftmargin=*]
        \item Developed responsive web pages for college of global sustainability using Bootstrap, HTML5, CSS3 and JavaScript.
    \end{itemize}
}

\resumeItemListEnd

\resumeSubheading
{Cognizant}{Chennai, TN, IN}
{Software Engineer}{Aug 2008 - Nov 2013}
\resumeItemListStart
\resumeItem{Performance Engineering}
{
    \begin{itemize}[leftmargin=*]
        \item Performed JVM profiling and heap dump analysis, to identify potential memory leaks and slow running code.
        \item Tuned JVM’s and recommended GC policies appropriate for the application, which improved the scalability and reduced memory footprint.
        \item Developed RESTful web services, providing in-house performance engineering tools as SAAS services for internal development teams.
    \end{itemize}
}

\resumeItemListEnd

\resumeSubHeadingListEnd

%-----------EDUCATION-----------------
\section{Education}
\resumeSubHeadingListStart
\resumeSubheading
{University of South Florida}{Tampa, FL, US}
{Master of Science in Management Information Systems;  GPA: 3.94/4.0}{Jan. 2014 -- May. 2015}
\resumeSubheading
{Madras Institute of Technology}{Chennai, TN, IN}
{Bachelor of Engineering in Computer Science;  GPA: 6.8/10.0}{Aug. 2004 -- July. 2008}
\resumeSubHeadingListEnd

%--------PROGRAMMING SKILLS------------
\section{Programming Skills}
\resumeSubHeadingListStart
\resumeSubItem{Languages}
{Go, Python, TypeScript, Scala, Java, C++}
\resumeSubItem{Frameworks}
{Node, Akka}
\resumeSubItem{Cloud Platforms}
{Kubernetes, RedHat OpenShift, IBM Cloud}
\resumeSubItem{Infrastructure}
{Chef, Ansible}
\resumeSubItem{Databases}
{RDBMS (PostgreSQL, MySQL), NoSQL (Redis, MongoDB)}
\resumeSubItem{ML}
{R, Tensorflow}
\resumeSubItem{Others}
{Containers, Git}
\resumeSubHeadingListEnd

%-----------PROJECTS-----------------
\section{Academic Projects}
\resumeSubHeadingListStart
\resumeSubItem{ICC Cricket Worldcup 2015 predictions}
{Collected and prepared match results data from espncricinfo.com.
    Built logistic regression models to estimate the probability of winning and evaluated against previous world cup.}
\resumeSubItem{Recommender Systems}
{Mentored by Dr.Balaji Padmanabhan, PhD. Investigated the problem of existing recommender algorithms used in businesses with hierarchical domains.
    Studied recommendation algorithms such as Probabilistic Inferences, SVD, CF and content filtering.}
\resumeSubItem{Predictive Models for P2P lending}
{Cleaned and analyzed a large XML (3.5 GB) data provided by peer-to-peer lending platform (www.prosper.com).
    Built predictive models for loans approval/rejections, loans default and risk calculation, and borrowers rating classification.}
\resumeSubItem{Digital Image Processing toolkit}
{Implemented image processing algorithms like Scaling, Edge detection, Fourier transforms and Hough Transforms in C++ (without built-in libraries like OpenCV)}
\resumeSubItem{Character Recognition toolkit}
{Implemented a classifier (minimum distance, Bayes moments and Nearest neighbor) for MNIST digit dataset using the central moments and covariance for each class as features.}
\resumeSubHeadingListEnd


%-------------------------------------------
\end{document}
